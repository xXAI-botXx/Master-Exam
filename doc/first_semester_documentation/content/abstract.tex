\chapter*{Abstract}
\label{cha:abstract}
	This work investigates complex paired image-to-image translation of the physical process of sound propagation. In more detail, the goal is to improve performance on the Physgen Benchmark, where \citeauthor{martin_spitznagel_physicsgen_2025} has already shown that there is still room for improvement in the complex domain of physical simulations using generative models.\\
	Several experiments are conducted to reduce the complexity of the task by modifying the input and target distributions. While such changes improve the predictions, they still result in inaccuracies. We also propose experiments to stabilize the adversarial loss and extend the loss function beyond L1 loss by introducing different loss terms with varying weights. These experiments showed that WGAN-GP harms accuracy, while the SSIM loss leads to improved performance.\\
	Lastly, different architectures, such as DepthAnythingV2 and our own developed model "HexaWaveNet", were tested. Neither met expectations. Additionally, a promising residual design was tested, which separates the reflections of the sound. This design could lead to SOTA results in the future but was hindered by the complex nature of predicting reflections in this work, which was further investigated.\\
	While this work provides novel insights into improving the learning of complex physical relationships, many open questions and proposed experiments remain. These include an improved version of the residual design introduced here, the use of the foundation model PhysiX/Cosmos-Predict2, scaling strategies, and experiments with transformer architectures.
	