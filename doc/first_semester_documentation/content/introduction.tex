\chapter{Introduction}
\label{cha:intro}
	
	
	
	\section{Problem Definition}
	\label{sec:intro-high_project}
		Geothermal energy plays an important role in the future of renewable energy and provides consistent energy compared to wind power and solar power \cite{geothermal_energy}\cite{herrenknecht_flyer}.\\
		To install systems that utilize geothermal energy, deep drilling is necessary in urban areas. Additionally, drilling at night is required, which creates an issue with noise pollution from the drilling system \cite{geothermal_energy}.\\
		The solution for this can lie in an AI-based monitoring system, predicting the current noise pollution based on the sound propagation. In addition to this, another connected system can then predict optimal drill settings for efficient yet silent drill parameter settings. Despite a current standing prototype, it is still challenging to learn the complex relationships and properties of physics from sound propagation. Therefore, this project aims to further investigate how to improve the learning of such complex relationships. To evaluate performance, we compare our results against the Physgen benchmark.
	
	
	
	\section{Current State of the Project}
	\label{sec:intro-problem}
		The Physgen benchmark is an image-to-image translation dataset with the results of multiple models available and it serves as the central focus of this project. It uses satellite images, preprocessed so that black pixels are buildings and white pixels are empty spaces, as input and the sound propagation is the output generated by a simulation software \cite{spitznagel_urban_2024-1}\cite{martin_spitznagel_physicsgen_2025}. There are 10,000 observations and a test dataset used as a benchmark, with results available from the simulation itself, as well as from models such as convAE \cite{dong2017learningdeeprepresentationsusing}, VAE \cite{Kingma_2019}, U-Net \cite{ronneberger_u-net_2015}, Pix2Pix \cite{isola_image--image_2016}, DDPM \cite{DBLP:journals/corr/abs-2006-11239}, SD \cite{DBLP:journals/corr/abs-2112-10752}, and DDBM \cite{zhou2023denoisingdiffusionbridgemodels}. Additionally, 4 variations determine the complexity of the ground truth data:
		\begin{itemize}[itemsep=1mm, parsep=0pt]
			\item Base sound simulation
			\item Base simulation with reflections (the sound reflects from buildings)
			\item Base simulation with diffractions (the sound bending around buildings)
			\item Base simulation with reflections and diffractions
		\end{itemize}
		
		\clearpage
		Currently, the best models on the Physgen benchmark are a pix2pix model, a diffusion model (with the downside of the processing), and a Normalizing Flow approach (Full Glow) \cite{martin_spitznagel_physicsgen_2025}\cite{achim_eckerle_evaluierung_2025}.\\
		This project focuses on the pix2pix architecture as a base approach for image-to-image translation because of the short inference and the short train time compared to the diffusion and the normalizing flow approach.\\
		\\
		While the base propagation and the diffraction already achieve high precision, the reflection is still a hard-to-solve task though this project concentrates on the reflection task. \\
		The baseline accuracies on the Physgen benchmark are shown in table \ref{tab:performance_base}.
		
		\begin{table}[h!]
			\centering
			\begin{tabular}{|l|c|c|c|c|}
				\hline
				\textbf{Model} & \textbf{LoS MAE} & \textbf{NLoS MAE} & \textbf{LoS wMAPE} & \textbf{NLoS wMAPE} \\
				\hline
				Pix2Pix & 2.14 & 4.79 & 11.30 & 30.67 \\
				DDBM & \textbf{1.93} & 6.38 & 18.34 & 79.13 \\
				Full Glow & 2.06 & \textbf{3.64} & \textbf{8.98} & \textbf{22.69} \\
				\hline
			\end{tabular}
			\caption{Performance comparison of base models on Physgen Reflection Test Dataset.}
			\label{tab:performance_base}
		\end{table}
		\FloatBarrier
		
		LoS refers to Line of Sight errors, which are all points on the image which are not behind a building in perspective of the middle point. Therefore NLoS refers to all pixels that are not in the line of sight of the middle point and are most likely the difficult part with complex physics (reflection and diffraction).
		
		% Through the experiments of this project, it became clear that one problem is that the reflection is only a slight change in value but complex bouncing behavior, it therefore makes sense that models do not learn them to keep the loss low. This is still a big challenge when separating the reflection as covered in section \ref{sec:experiments-residual_architecture}.
		
		
		
	\section{Goal}
	\label{sec:intro-goal}
		%This project tries to get new insights into the effects of different approaches on the learning of complex relationships with generative models, with the focus on improvement also 
		%The project aims to collect new insights for different approaches to the problem. These approaches lie in the usage of novel architectures and changes to the input, target and learning behavior of the models. Concretely, this means
		%\\
		%As comparison this project will refer to the results in table \ref{tab:performance_base} on the reflection variation of Physgen benchmark from \cite{martin_spitznagel_physicsgen_2025}\cite{achim_eckerle_evaluierung_2025}.
		
		The project aims to collect new insights into how different approaches affect the learning of complex physical relationships using generative models. These approaches include the use of novel architectures, as well as changes to the input, target, and training behavior of the models. Concretely, this means experimenting with various architectural components, such as residual connections and alternative encoder-decoder designs, modifying the input data distribution to reduce ambiguity, and enhancing the loss functions to better emphasize difficult aspects like reflections.\\
		One particular challenge is that reflections, while visually subtle, result from complex bouncing behavior and are thus hard to model. Many existing models fail to learn reflections effectively, as they tend to prioritize reducing the overall loss—often ignoring small but semantically important deviations. This project focuses on tackling exactly this issue by isolating the reflection component and explicitly testing new model behaviors on it.\\
		%In addition to improving model performance on the reflection task, the project also aims to contribute more broadly to the understanding of how generative models can learn fine-grained physical phenomena. A long-term vision is to support the development of real-time, AI-driven systems for monitoring and optimizing noise propagation in urban drilling scenarios, such as those required for geothermal energy extraction.
		As a point of comparison, this project refers to the baseline results shown in table \ref{tab:performance_base} for the reflection variation of the Physgen benchmark, as presented in \cite{martin_spitznagel_physicsgen_2025}\cite{achim_eckerle_evaluierung_2025}.\\
		The goal of this project is not only to achieve new state-of-the-art results on the Physgen benchmark, but also to obtain physically accurate predictions in complex scenarios—particularly in modeling reflections and their interactions with surrounding buildings.
		
		
		
	\section{Structure of the work}
	\label{sec:structure-of-work}
		In order to tackle the challenges of learning complex physical relationships, this project explores a variety of approaches structured into different experimental directions.\\
		Novel architectures are invented and tested in section \ref{sec:experiments-generator_architecture_exploration}.\\
		Splitting the target distribution into smaller (less complex) distributions is another approach that will be tested in different ways. The learning of the residuals still is very challenging and needs much engineering like combined loss, weighted loss, and masked loss to work. See section \ref{sec:experiments-residual_architecture} and \ref{sec:experiments-only_reflections_with_few_buildings}.\\
		In section \ref{sec:experiments-stabilized_adversarial_loss} a more stabilized adversarial loss is applied to investigate if this could help in learning the complex distribution.\\
		Additionally in section \ref{sec:experiments-input-abstraction} the input distribution is changed, so that the base propagation becomes the input of the model, to see if the transfer to the complex output distribution is then more precise possible.
		
		
		
		
		
		
		
		