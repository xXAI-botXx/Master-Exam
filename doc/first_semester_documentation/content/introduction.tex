\chapter{Introduction}
\label{cha:intro}
	
	
	
	\section{High-Level Project}
	\label{sec:intro-high_project}
		Geothermal energy plays an important role in the future of renewable energy and provides consistent energy compared to wind power and solar power \cite{geothermal_energy}\cite{herrenknecht_flyer}.\\
		To install systems that utilize geothermal energy, deep drilling is necessary in urban areas. Additionally, drilling at night is required, which creates an issue with noise pollution from the drilling system \cite{geothermal_energy}.\\
		A solution is an AI for monitoring current noise pollution by predicting sound propagation, as well as a drill-setting prediction for efficient yet silent drill parameter settings, as described in Section \ref{sec:lit-core}. However, it is still challenging to learn the complex relationships and properties of physics from sound propagation and so this project aims to investigate more in how to improve the learning of such complex relationships.\\
		"Physgen" an image-to-image translation dataset is the central focus of this project. It uses satellite images, preprocessed so that black pixels are buildings and white pixels are empty spaces, as input and the sound propagation is the output generated by a simulation software \cite{spitznagel_urban_2024-1}\cite{martin_spitznagel_physicsgen_2025}. There are 10k observations and a test dataset as a benchmark with results from the simulation itself, convAE, VAE, UNet, Pix2Pix, DDPM, SD, and DDBM. Additionally, 4 variations determine the complexity of the ground truth data:
		\begin{itemize}[itemsep=1mm, parsep=0pt]
			\item Base sound simulation
			\item Base simulation with reflections (the sound reflects from buildings)
			\item Base simulation with diffractions (the sound bending around buildings)
			\item Base simulation with reflections and diffractions
		\end{itemize}
	
	
	
	\section{Problem State}
	\label{sec:intro-problem}
		Currently, the best models on the Physgen benchmark are a pix2pix model, a diffusion model (with the downside of the processing), and a Normalizing Flow approach (Full Glow) \cite{martin_spitznagel_physicsgen_2025}\cite{achim_eckerle_evaluierung_2025}.\\
		This project focuses on the pix2pix architecture as a base approach for image-to-image translation because of the short inference and the short train time compared to the diffusion and the normalizing flow approach.\\
		\\
		While the base propagation and the diffraction already achieve high precision, the reflection is still a hard-to-solve task though this project concentrates on the reflection task. \\
		Through the experiments of this project, it became clear that one problem is that the reflection is only a slight change in value but complex bouncing behavior, it therefore makes sense that models do not learn them to keep the loss low. This is still a big challenge when separating the reflection as covered in section \ref{sec:experiments-residual_architecture}.
		
		
		
	\section{Goal}
	\label{sec:intro-goal}
		This project tries to get new insights into the effects of different approaches on the learning of complex relationships with generative models. With the focus on improvement also novel architectures are invented and tested in section \ref{sec:experiments-generator_architecture_exploration}.\\
		Splitting the distribution into smaller (less complex) distributions is another approach that will be tested in different ways. The learning of the residuals still is very challenging and needs much engineering like combined loss, weighted loss, and masked loss to work. See section \ref{sec:experiments-residual_architecture} and \ref{sec:experiments-only_reflections_with_few_buildings}.\\
		In section \ref{sec:experiments-stabilized_adversarial_loss} a more stabilized adversarial loss is applied to investigate if this could help in learning the complex distribution.\\
		Additionally in section \ref{sec:experiments-input-abstraction} the input distribution is changed (base propagation as input) to see if the transfer to the complex output distribution is then more precise possible.
		\\
		\\
		As comparison this project will refer to the results in table \ref{tab:performance_base} on the reflection variation of Physgen benchmark from \cite{martin_spitznagel_physicsgen_2025}\cite{achim_eckerle_evaluierung_2025}:
		
		\begin{table}[h!]
			\centering
			\begin{tabular}{|l|c|c|c|c|}
				\hline
				\textbf{Model} & \textbf{LoS MAE} & \textbf{NLoS MAE} & \textbf{LoS wMAPE} & \textbf{NLoS wMAPE} \\
				\hline
				Pix2Pix & 2.14 & 4.79 & 11.30 & 30.67 \\
				DDBM & \textbf{1.93} & 6.38 & 18.34 & 79.13 \\
				Full Glow & 2.06 & \textbf{3.64} & \textbf{8.98} & \textbf{22.69} \\
				\hline
			\end{tabular}
			\caption{Performance comparison of base models on Physgen Reflection Test Dataset.}
			\label{tab:performance_base}
		\end{table}
		
		LoS refers to Line of Sight errors, which are all points on the image which are not behind a building in perspective of the middle point. Therefore NLoS refers to all pixels that are not in the line of sight of the middle point and are most likely the difficult part with complex physics (reflection and diffraction).
		
		
		
		
		
		
		
		