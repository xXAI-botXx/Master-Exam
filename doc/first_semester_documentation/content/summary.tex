\chapter{Summary}
\label{cha:summary}
	
	
	\section{Achievements}
	\label{sec:sum-reached}
		This work proposes many experiments towards improvement on the Physgen benchmark and most likely focuses on the reflection variation.\\
		This work shows that the base-simulation as input helps to achieve higher accuracies, but it is not enough to learn the complex propagation of reflections. Another outcome is that the models quickly (about 50 epochs) find a local minimum which can't be escaped easily.\\
		Furthermore, stabilizing the adversarial loss through Wasserstein loss and gradient penalty does not provide improvements, but rather leads to worse predictions, and so should not be used in future experiments.\\
		Other architectures failed to achieve SOTA results, but that does not mean that there could not be another better fitting architecture, and it is possible that a data scaling or an architectural scaling could improve these other architectures. \\
		The proposed residual design with the prediction of only the reflection did not lead to SOTA results yet, but after the part-wise successful engineering of the reflection-only prediction, this approach could lead to new SOTA results.\\
		Overall, this project did not reach the end of improvements, but it has already collected valuable insights about which work and where are still challenges.
	
	\section{Future}
	\label{sec:sum-future}
		In the future, there are multiple potential experiments with promising outcomes.\\
		First, the testing of the improved reflection-only predictions in the residual design should have high priority, though this experiment has already been conducted; this time, the loss weights are improved.\\
		Another similar experiment would be to test the residual design by changing the input of the reflection-only from satellite image to the output of the other model, the prediction of the base-sound propagation, in the residual design. As experiment \ref{sec:experiments-input-abstraction} already showed improvement when not predicting end-to-end.\\
		There are also other architectures to test. For example, the foundation model for physics if the checkpoints are available, or maybe the base-model: Cosmos \cite{nvidia2025cosmosworldfoundationmodel}.
		\clearpage
		Many architectures have already been investigated, but transformer-based models have not been well covered yet. Therefore, an investigation into transformer-based architectures could also be promising.\\
		Continuing with the current experiments would also mean trying to break down the complexity even further, for example, into angles of the reflections. \\
		The experiments also hint that another loss could lead to massive improvements. It could make sense to integrate a physics-based loss to validate the physical correctness and find a local minimum closer to the global minimum.\\
		Lastly, two scaling experiments could also be interesting. One where data scaling and one where the scaling of architecture is investigated.\\
		To summarize, there are many experiments for improvement left.
	
	
	